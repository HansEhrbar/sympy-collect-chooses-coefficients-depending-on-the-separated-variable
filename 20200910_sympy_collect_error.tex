\documentclass[legalpaper]{article}
\usepackage{amssymb,amsmath}
\usepackage[dvipsnames]{xcolor}
\newcommand\mycdot{\,}
\usepackage{listings}


\newcommand\alphaC{{\color{OliveGreen}\mathbf\alpha}}
\newcommand\betaC{{\color{red}\mathbf\beta}}
\newcommand\gammaC{{\color{blue}\mathbf\gamma}}
\newcommand\psiC{{\color{OliveGreen}\mathbf\psi}}
\newcommand\thetaC{{\color{red}\mathbf\theta}}
\newcommand\CCCtwoC{{\color{red}\mathbf{C_2}}}
\newcommand\tauC{{\color{red}\mathbf\tau}}
\newcommand\alphaSquaredC{{\color{OliveGreen}\mathbf{\alpha^2}}}
\newcommand\betaSquaredC{{\color{red}\mathbf{\beta^2}}}
\newcommand\gammaSquaredC{{\color{blue}\mathbf{\gamma^2}}}
\newcommand\tauSquaredC{{\color{red}\mathbf{\tau^2}}}
\newcommand\thetaSquaredC{{\color{red}\mathbf{\theta^2}}}

\begin{document}
  Here is a polynomial for which I need a mathematical proof that it is non-negative:
    \begin{align}
    \CCCtwoC&= \bigl(\alphaC{\mycdot}\psiC+\betaC{\mycdot}\thetaC+\gammaC{\mycdot}(\thetaC+\psiC)\bigr)^2
     -4{\mycdot}(\alphaC+\betaC+\gammaC){\mycdot}\gammaC{\mycdot}\psiC{\mycdot}\thetaC\label{CCC_2}
    \end{align}
 Through some tedious math I can show that \(\CCCtwoC\) satisfies the identity
  \begin{align}
    (\alphaC+\gammaC)^2{\mycdot}\frac{\CCCtwoC}{\thetaSquaredC}
    &=4{\mycdot}\alphaC{\mycdot}\betaC{\mycdot}\gammaC
    {\mycdot}(\alphaC+\betaC+\gammaC)
    +\Bigl((\alphaC+\gammaC)^2{\mycdot}\frac{\psiC}{\thetaC}+\alphaC{\mycdot}\betaC-(\alphaC+\betaC+\gammaC){\mycdot}\gammaC\Bigr)^2\label{positive_version_of_CCC_2}
  \end{align}
  Since the rhs is obviously non-negative this is the proof I need.  
  The proof introduces the new parameter \(\tauC=\psiC/\thetaC\) and shows that \(\CCCtwoC/\thetaSquaredC\) can be written in the form
            \begin{align}
      \frac{\CCCtwoC}{\thetaSquaredC}&= {A}_0 +  {A}_1{\mycdot}\tauC + {A}_2{\mycdot}\tauSquaredC\notag\\
     \intertext{with}
       {A}_0 &= (\betaC+\gammaC)^2\label{EEE_0}\\
       {A}_1 &= 2{\mycdot}\bigl(\alphaC{\mycdot}\betaC-(\alphaC+\betaC+\gammaC){\mycdot}\gammaC\bigr) \label{EEE_1}\\
       {A}_2 &= (\alphaC+\gammaC)^2 \label{EEE_2}\\
       \intertext{and re-writes it in the form}
       \frac{\CCCtwoC}{\thetaSquaredC}&= {E} +  {F}{\mycdot}(\tauC-{G})^2\notag
            \end{align}
            where all three polynomials \({E}\), \({F}\), and \({G}\) do not contain \(\tauC\).
            This gives \eqref{positive_version_of_CCC_2}.

                 I tried to use \texttt{SymPy} for this proof with the \texttt{python3} code
                 \texttt{20200910\_sympy.py}
                 submitted at the same time.
 \begin{scriptsize}
    \lstinputlisting[language=python]{20200910_sympy.py}
 \end{scriptsize}
 The output of the \texttt{sympy.pprint} command is
  \begin{align}
    \CCCtwoC &=-\gammaC{\mycdot}\tauC{\mycdot}\thetaSquaredC{\mycdot}(4{\mycdot}\alphaC + 4{\mycdot}\betaC + 4{\mycdot}\gammaC)
    + (\alphaC{\mycdot}\tauC{\mycdot}\thetaC + \betaC{\mycdot}\thetaC + \gammaC{\mycdot}(\tauC{\mycdot}\thetaC + \thetaC))^2\notag
  \end{align}
 which can be re-written as
  \begin{align}
    \CCCtwoC &=-4{\mycdot}\gammaC{\mycdot}\tauC{\mycdot}\thetaSquaredC{\mycdot}(\alphaC + \betaC + \gammaC)
    + \thetaSquaredC{\mycdot}\bigl((\alphaC+\gammaC){\mycdot}\tauC + \betaC + \gammaC)^2\notag
  \end{align}
  This is similar to \eqref{positive_version_of_CCC_2} but has one important difference:
  It can be re-written in the form
      \begin{align}
      \CCCtwoC&= {E} +  {F}{\mycdot}(\tauC-{G})^2\notag
      \end{align}
      but whereas the decomposition in my proof required that all three polynomials \({E}\), \({F}\), and \({G}\)
      do not contain \(\tauC\),
      the above \({E}=-4{\mycdot}\gammaC{\mycdot}\tauC{\mycdot}\thetaSquaredC{\mycdot}(\alphaC + \betaC + \gammaC)\)
      does contain \(\tauC\).
      I think this is a design error in the \texttt{sympy.collect} command.
      Can someone make a \texttt{sympy.collect\_unrelated} command
      which does this?

\end{document}

